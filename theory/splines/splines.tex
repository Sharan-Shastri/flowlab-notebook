\documentclass[nofonts]{tufte-book}

% ---- packages ----
\usepackage{ebgaramond} % Use Garamond Font
\usepackage{booktabs} % Better horizontal rules in tables
\usepackage{nameref}  % reference sections by name and not just number
\usepackage{graphicx} % Needed to insert images into the document
\usepackage{amsmath} % package for math stuff
\usepackage{listings}  % for code
\usepackage[usenames,dvipsnames]{xcolor}  % colors

% --- additional setup ----
\hypersetup{colorlinks} 
\graphicspath{{figures/}} % Sets the default location of figures

%%
%% Julia definition (c) 2014 Jubobs
%%
\lstdefinelanguage{Julia}%
{morekeywords={abstract,break,case,catch,const,continue,do,else,elseif,%
		end,export,false,for,function,immutable,import,importall,if,in,%
		macro,module,otherwise,quote,return,switch,true,try,type,typealias,%
		using,while},%
	sensitive=true,%
	alsoother={\$},%
	morecomment=[l]\#,%
	morecomment=[n]{\#=}{=\#},%
	morestring=[s]{"}{"},%
	morestring=[m]{'}{'},%
}[keywords,comments,strings]%

\lstset{%
	language         = Julia,
	basicstyle       = \footnotesize,
	keywordstyle     = \bfseries\color{blue},
	stringstyle      = \color{magenta},
	commentstyle     = \color{ForestGreen},
	showstringspaces = false,
	breaklines=true,
}

% ----- new commands -------



% --- BOOK META-INFORMATION -----

\title{FLOW Lab Notebook}
\author[]{The BYU FLOW Lab}
\publisher{}



\usepackage{amsthm,amssymb}
\begin{document}
	
\chapter*{Splines}
\chapterauthor{Judd Mehr}
\label{ch:splines}


\section{Introduction}
\label{sec:splinesintro}

Splines are useful in many applications including, but not limited to: Engineering, Architecture, Graphic Design, and Mathematics. Splines began as physical, rather than mathematical objects. In the ship building, and later the aircraft building, industry, as well as other architectural applications, drafters would use thin pieces of wood or metal called splines that they would bend through key points to create smooth interpolative curves.  Today, splines are used much in the same way, albeit primarily in virtual applications.  In engineering, splines are most often used for data interpolation and graphic design/representation (CAD). Simply put, splines are piecewise polynomial functions; thus they behave more or less as do polynomials, but have much greater flexibility and are therefore often preferred for accuracy in interpolation and flexibility in design. There are many kinds of splines, but here we discuss Bézier Curves, B-Splines, and NURBS.

\section{Bézier Curves}
\subsection{Understanding the Parametric Nature of Bézier Curves}
Bézeir curves are parametric.  To understand this concept and begin building intuition behind general spline structure, we will jump right in and create a basic quadratic Bézier curve. We start with the parametric function $P(t)$, defined as:

\begin{equation}
\label{eq:parametricrelation}
P(t) = (1-t)P_0 + tP_1
\end{equation}

where $t$ is a parameter such that $0 \leq t \leq 1$, and $P_{(\cdot)}$ are 2D control points (note that they are 2D in this example, but are not, in general, limited to 2 dimensions). Notice that when $t=0$ we are at point $P_0$ and when $t=1$ we are at point $P_1$. 

\begin{figure}[htbp]
	\centering
	\includegraphics[width=3.0in]{para1.pdf}
	\caption{Primary Parametric Curve.}
	\label{fig:para1}
\end{figure}

Next, let's define some parametric points $Q_{(\cdot)}$ as:

\begin{align}
Q_0(t) &=  (1-t)P_0 + tP_1 \\
Q_1(t) &=  (1-t)P_1 + tP_2
 \end{align} 
 
Taking the points $Q_{(\cdot)}$ and applying the same equation \cref{eq:parametricrelation}, we can recursively create another parametric curve. Several examples are shown in figure \cref{fig:para2} as we proceed through the range of $t$. If we apply equation \cref{eq:parametricrelation} once more, the connected points create the quadratic curve, $C(t)$

\begin{align}
C(t) &= (1-t)Q_0 + tQ_1 \\
&= (1-t)^2P_0 + 2t(1-t)P_1+t^2P_2 
\end{align} 

as shown in figure \cref{fig:para3}. This recursive application of parametric polynomials is how Bézier curves are defined.
 
 \begin{figure}[htbp]
 	\centering
 	\includegraphics[width=\textwidth]{para2.pdf}
 	\caption{Recursively Defined Parametric Curves.}
 	\label{fig:para2}
 \end{figure}

 
\begin{figure}[htbp]
	\centering
	\includegraphics[width=5in]{para3.pdf}
	\caption{Recursively Defined Quadratic Bézier.}
	\label{fig:para3}
\end{figure}


\subsection{Bernstein Polynomials}
Before giving a more concise, more easily implemented defintion for Bézier Curves, let us take a minute to undertand Bernstein polynomials; which are defined as

\begin{equation} B_{i,n}(t) = {n\choose i} t^i (1-t)^{n-i} ~~~~~0\leq t \leq1 \end{equation}

where \begin{equation} {n\choose i} = \frac{n!}{i!(n-i)!} \end{equation} 

\begin{figure}[htbp]
	\centering
	\includegraphics[width=3.0in]{bernstein.pdf}
	\caption{Bernstein polynomials for $n=4$.}
	\label{fig:bernstein}
\end{figure}

From figure \cref{fig:bernstein} we can see some important properties of Berstein polynomials. We won't state why they are important just yet, but it will be good to recognize them now.

\begin{enumerate}
	\item $B_{i,n}(t) > 0$ everywhere in the range of $t$.
	\item $\sum_{i=0}^n B_{i,n}(t) = 1$ everywhere in the range of $t$.
	\item $B_{i,n}(t)$ has only one maximum in the range of $t$, at $t=i/n$.
	\item The set of  $B_{i,n}(t)$ are symmetric about $t=1/2$.
\end{enumerate}

\subsection{Bézier Curve Definition Based on Bernstein Polynomials}

Bézier curves can be expressed as follows.

\begin{equation} \textbf{C}(t) = \sum^n_{i=0}B_{i,n}(t) \textbf{P}_i~~~~~0\leq t \leq1 \end{equation}

where $B_{i,n}(t)$ are the basis (sometimes called blending or shape) functions. For Bézier curves, these basis functions are indeed Bernstein polynomials. \(\textbf{P}_i\) are geometric control points.  This formulation (mathematically equivalent to polynomial power basis form derived in like manner to the example of a quadratic curve given above) for Bézier curves is attractive due to its well behaved nature in numerical applications. Furthermore, it tends to be relatively intuitive to work with.

\section{Basis Splines (B-Splines)}

\subsection{B-Spline Basis Functions}
B-Spline construction revolves basis functions, and these basis functions are dependent on knots, or the junctions of the Bézier curves which make up the B-Spline. To define a B-Spline basis and later a B-Spline curve, we first begin by defining a knot vector, $\mathcal{T} = \{t_1, t_2  \ldots t_{m}\}$, where the knots $t_i$ are non-decreasing.  (The full importance of the knot vector will be clear later.) Using this knot vector we define $N_{i,p}(t)$, the $i$th basis function of degree $p$ (note that degree and order are not synonymous when speaking of splines; in general, order = $p+1$) using the Cox-deBoor algorithm:

\begin{equation} 
\label{eqn:basis0}
N_{i,0}(t) = 
\begin{cases} 
1 & \text{if  } t_i \leq t < t_{i+1} \\
0 & \text{otherwise}
\end{cases} 
\end{equation}

\begin{equation}
\label{eqn:basisgeneral} 
N_{i,p}(t) = \frac{t-t_i}{t_{i+p}-t_i} N_{i,p-1}(t) + \frac{t_{i+p+1}-t}{t_{i+p+1}-t_{i+1}}N_{i+1,p-1}(t) \end{equation}

\bigskip

This algorithm can be visualized as a triangular scheme:
\begin{equation}
\begin{matrix}
	N_{0,0}(t) & \rightarrow & N_{0,1}(t) & \cdots & N_{0,p-1}(t) & \rightarrow & N_{0,p}(t)\\ 
	& \nearrow &  &  &  & \nearrow & \\ 
	N_{1,0}(t) & \rightarrow & N_{1,1}(t) & \cdots & N_{1,p-1}(t) &  & \\ 
	\vdots &  & \vdots &  &  &  & \\ 
	N_{n-1,0}(t) & \rightarrow & N_{n-1,1}(t) &  &  &  & \\ 
	& \nearrow &  &  &  &  & \\ 
	N_{n,0}(t) &  &  &  &  &  & \\ 
\end{matrix}
\end{equation}

The derivative of a basis function, $N_{i,p}(t)$, is given by

\begin{equation}
\label{eqn:basisderivative}
N_{i,p}'(t) = \frac{p}{t_{i+p} - t_i} N_{i,p-1}(t) - \frac{p}{t_{i+p+1} - t_{i+1}} N_{i+1,p-1}(t)
\end{equation}

(See \cref{sec:basisderivativeproof} for the proof for this.) In general, the $k$th derivative is given by

\begin{equation}
\label{eqn:basisder1}
	N_{i,p}^{(k)}(t) = p \left( \frac{N_{i,p-1}^{(k-1)}(t)}{t_{i+p} - t_i} - \frac{N_{i+1,p-1}^{(k-1)}(t)}{t_{i+p+1} - t_{i+1}} \right)
\end{equation}

or in terms of $N_{i,p-k}(t), \ldots, N_{i+k,p-k}(t)$ rather than derivatives:

\begin{equation}
\label{eqn:basisder2}
	N_{i,p}^{(k)}(t) = \frac{p!}{(p-k)!} \sum^k_{j=0} a_{k,j} N_{i+j,p-k}(t
\end{equation}

with

\begin{align}
a_{0,0} &= 1\\
a_{k,0} &= \frac{a_{k-1,0}}{t_{i+p-k+1} - t_i}\\
a_{k,j} &= \frac{a_{k-1,j} - a_{k-1,j-1}}{t_{i+p+j-k+1} - t_{i+j} } ~~~ j=1,\ldots,k-1\\
a_{k,k} &= \frac{-a_{k-1,k-1}}{t_{i+p+1} - t_{i+k} }
\end{align}


\subsection{B-Spline Definition}
As mentioned, a B-Spline is a series of Bézier curves joined end to end with junctions called knots.  The definition of a B-Spline is

\begin{equation} \textbf{C}(t) = \sum^n_{i=0}N_{i,p}(t) \textbf{P}_i~~~~~a\leq t \leq b \end{equation}

and has derivatives

\begin{equation} \textbf{C}^{(k)}(t) = \sum^n_{i=0}N^{(k)}_{i,p}(t) \textbf{P}_i~~~~~a\leq t \leq b \end{equation}

where $\textbf{P}_i$ are control points and $N_{i,p}(t) $ are the basis functions previously defined. A $p$th degree B-Spline has $n+1$ basis functions, where $n = m-p-1$, and the number of knots, $t_i$, in the knot vector, $\mathcal{T}$, over which the spline is defined, is $m+1$ and the first and last knots are repeated $p+1$ times.
 
\begin{equation} \mathcal{T} = \{\underbrace{a,\ldots,a}_{p+1},t_{p+1},\ldots,t_{m-p-1},\underbrace{b,\ldots,b}_{p+1} \}  \end{equation}

%TODO add example of Bspline showing bases and how the knot vector affects them, as well as the associated curves.  p=2 and p=3 are probably sufficient.


\section{Non-uniform Rational B-Splines (NURBS)}
Now that we have a basic understanding of B-Splines, we can move on to NURBS, which are just a superset of B-Splines, but are therefore slightly  more useful. We will look at a few useful things we can do with splines, applied to NURBS.

\subsection{NURBS Definition}
NURBS can be thought of as weighted B-Splines, and are defined as follows:

\begin{equation}\label{eqn:nurbsdef} \textbf{C}(t) = \frac{ \displaystyle \sum^n_{i=0}N_{i,p}(t) w_i \textbf{P}_i }{ \displaystyle \sum^n_{i=0}N_{i,p}(t) w_i }~~~~~a\leq t \leq b \end{equation}  

Where again $\textbf{P}_i$ are the control points, and $\{w_i\}$ are the weights and the knot vector, $\mathcal{T}$ is defined as before except that the knots specifically need \textit{not} be uniform.

To simplify this expression, we can define rational basis functions

\begin{equation} R_{i,p}(t) = \frac{ N_{i,p}(t) w_i } { \displaystyle \sum^n_{i=0}N_{i,p}(t) w_i }\end{equation}  

and our expression becomes

\begin{equation} \textbf{C}(t) = \sum^n_{i=0} R_{i,p}(t) \textbf{P}_i  \end{equation}  

%TODO Add an example here showing the fitting of a quarter circle.

We can also describe an equivalent, but nonrational (piecewise B-Spline) curve by applying the weights to the control points through the following projection

\begin{equation}
	P^w_i = [x_iw_i,y_iw_i,z_iw_i,w_i]
\end{equation}

This gives us the nonrational curve

\begin{equation}
	\textbf{C}^w(t) = \sum^n_{i=0}N_{i,p}(t) \textbf{P}^w_i
\end{equation}

Both \(\textbf{C}(t)\) and \(\textbf{C}^w(t)\) are typically referred to as NURBS curves, despite \(\textbf{C}^w(t)\) not actually being rational.

\subsection{NURBS Derivatives}
In order to make things easier, we want to find the derivatives of NURBS curves in terms of the nonrational form. The procedure proceeds as follows.

Let 

\begin{equation}
	\textbf{C}^w(t) = [\textbf{A}(t),~w(t)]
\end{equation}

where \(\textbf{A}(t)\) is the vector of the first three elements, and \(w(t)\) is the final element, of \(\textbf{C}^w(t)\). Next we take \(\textbf{C}(t)\) and multiply by 1, and put it in terms of \(\textbf{A}(t)\) and \(w(t)\), noting that \(w(t)=\sum^n_{i=0}N_{i,p}(t) w_i \)

\begin{equation}
\label{eqn:nonrationalequiv}
	\textbf{C}(t) = \frac{ w(t) \textbf{C}(t) }{ w(t) } = \frac{ \textbf{A}(t) }{ w(t) }
\end{equation}

we can thus see that \(\textbf{A}(t)\) is also the numerator of equation \cref{eqn:nurbsdef}.

Taking the first derivative of $\textbf{C}'(t)$ via the quotient rule yields

\begin{equation}
	\textbf{C}'(t) = \frac{ w(t)\textbf{A}'(t) - w'(t) \textbf{A}(t) }{ w(t)^2 }
\end{equation}

substituting in for \(\textbf{A}(t)\) from the relation from \cref{eqn:nonrationalequiv}
\begin{equation} 
	\textbf{C}'(t) = \frac{  w(t)\textbf{A}'(t) - w'(t) \textbf{C}(t)w(t) }{ w(t)^2 }
\end{equation}

then simplifying gives us the expression for the first derivative of \(\textbf{C}(t) \):
\begin{equation} 
\textbf{C}'(t) = \frac{ \textbf{A}'(t) - w'(t) \textbf{C}(t) }{ w(t) }
\end{equation}

For higher order derivatives, we apply the generalized chain rule to \(\textbf{A}(t)\)

\begin{equation} 
\textbf{A}^{(k)}(t) = (w(t)\textbf{C}(t))^{(k)} = \displaystyle \sum^k_{i=0}\binom{k}i w^{(i)} (t) \textbf{C}^{(k-i)} (t) 
\end{equation}

taking out the first element in the summand (where $i=0$)

\begin{equation} 
\textbf{A}^{(k)}(t) = w(t)\textbf{C}^{(k)}(t) + \displaystyle \sum^k_{i=1}\binom{k}i w^{(i)} (t) \textbf{C}^{(k-i)} (t) 
\end{equation}

and then rearranging for \(\textbf{C}^{(k)}(t)\), we arrive at the expression for the $k$th derivative of the NURBS curve.

\begin{equation} 
	\textbf{C}^{(k)}(t) = \frac{ \textbf{A}^{(k)}(t) - \displaystyle \sum^k_{i=1} \binom{k}i w^{(i)}(t) \textbf{C}^{(k-i)} (t) }{ w(t) }
\end{equation}

The derivatives for \(\textbf{A}^{(k)}(t)\) and \(w^{(i)}(t)\) can be calculated by applying the $k$th derivatives of their basis functions (see \cref{eqn:basisder1} or \cref{eqn:basisder2}).

\subsection{Refinement}

\subsubsection{Bézier Extraction}
In order to understand the different kinds of spline refinement discussed below, we must first discuss Bézier extraction.


\subsubsection{Degree Elevation}
Degree elevation, sometimes called p-refinement, is a process by which a p-degree spline is represented exactly in a higher order space.

%TODO Show an example (the one in the docs is fine)

\subsubsection{Knot Insertion}
Knot insertion, sometimes called h-refinement, is a process by which extra knots are inserted into the knot vector, basically adding more Bézier curves to the spline where the knots are inserted, allowing for greater curve flexibility without increase the curve order.

%TODO Show an example (the one in the docs is fine)

%\section{Tensor Product Surfaces}

\bibliographystyle{aiaa} 
\bibliography{splines.bib}{}


\begin{appendices}
\renewcommand{\thesection}{Appendix \Alph{section}.}

\section{Proof by Induction for Basis Functions Derivatives}
\label{sec:basisderivativeproof}
The proof of \cref{eqn:basisderivative} comes from section 2.3 in reference \cite{Piegl1997The-NURBS-Book}. 

\renewcommand{\qedsymbol}{\rule{0.7em}{0.7em}}

\begin{proof}
Let \(p=1\). By \cref{eqn:basis0} \(N_{i,p-1}(t)\) and \(N_{i+1,p-1}(t)\) are either 0 or 1. Thus by \cref{eqn:basisgeneral}, for \(p=1\), \(N_{i,p}^{'}(t)\) is either 

\[ \frac{1}{ t_{i+1} - t_{i} } ~~\text{or}~~ \frac{-1}{ t_{i+2} - t_{i+1} } \]

Assume now that \cref{eqn:basisderivative} is true for \(p-1,~p>1\). Applying the product rule, we differentiate \cref{eqn:basisgeneral} to obtain

\begin{equation}
\begin{aligned}
N_{i,p}'(t) &= \frac{1}{t_{i+p} - t_i} N_{i,p-1}(t) + \frac{t-t_i}{t_{i+p}-t_i} N_{i,p-1}^{'}(t)\\
			&+ \frac{-1}{t_{i+p+1} - t_{i+1}} N_{i+1,p-1}(t) + \frac{t_{i+p+1}-t}{t_{i+p+1}-t_{i+1}}N_{i+1,p-1}^{'}(t)
\end{aligned}
\end{equation}

Now substitute in \cref{eqn:basisderivative} for \(N_{i,p-1}^{'}(t)\) and \(N_{i+1,p-1}^{'}(t)\)

\begin{equation}
\begin{split}
N_{i,p}'(t) =& \frac{1}{t_{i+p} - t_i} N_{i,p-1}(t) + \frac{-1}{t_{i+p+1} - t_{i+1}} N_{i+1,p-1}(t) \\ 
& + \frac{t-t_i}{t_{i+p}-t_i} \left( \frac{p-1}{t_{i+p-1} - t_i} N_{i,p-2}(t) - \frac{p-1}{t_{i+p} - t_{i+1}} N_{i+1,p-2}(t) \right) \\
&+ \frac{t_{i+p+1}-t}{t_{i+p+1}-t_{i+1}} \left( \frac{p-1}{t_{i+p} - t_{i+1}} N_{i+1,p-2}(t) - \frac{p-1}{t_{i+p+1} - t_{i+2}} N_{i+2,p-2}(t) \right)
\end{split}\raisetag{3\baselineskip}
\end{equation}

Gather like terms:

\begin{equation}
\begin{split}
\label{eqn:gatherterms}
N_{i,p}'(t) =& \frac{1}{t_{i+p} - t_i} N_{i,p-1}(t)\\
& + \frac{-1}{t_{i+p+1} - t_{i+1}} N_{i+1,p-1}(t) \\ 
& + \frac{p-1}{t_{i+p-1} - t_i} ~\frac{t-t_i}{t_{i+p}-t_i} N_{i,p-2}(t)\\
& + \frac{p-1}{t_{i+p} - t_{i+1}} \left( \frac{t_{i+p+1}-t}{t_{i+p+1}-t_{i+1}} - \frac{t-t_i}{t_{i+p}-t_i}\right)  N_{i+1,p-2}\\
& - \frac{p-1}{t_{i+p+1} - t_{i+2}}~\frac{t_{i+p+1}-t}{t_{i+p+1}-t_{i+1}} N_{i+2,p-2}(t)
\end{split}\raisetag{3\baselineskip}
\end{equation}

Add and subtract 1 from the terms in parenthesis, allowing us to put things in a convenient form shortly.

\begin{equation}
\begin{split}
\label{eqn:plusminus1}
\frac{t_{i+p+1}-t}{t_{i+p+1}-t_{i+1}} - \frac{t-t_i}{t_{i+p}-t_i} =& -1 + \frac{t_{i+p+1}-t}{t_{i+p+1}-t_{i+1}} +1 - \frac{t-t_i}{t_{i+p}-t_i}\\
	=& -\frac{t_{i+p+1} - t_{i+1}}{t_{i+p+1} - t_{i+1}} + \frac{t_{i+p+1}-t}{t_{i+p+1}-t_{i+1}}\\
	&+ \frac{t_{i+p} - t_{i}}{t_{i+p} - t_{i}} - \frac{t-t_i}{t_{i+p}-t_i}\\
	=& \frac{t_{i+p} - t}{t_{i+p} - t_{i}} - \frac{t - t_{i+1}}{t_{i+p+1} - t_{i+1}}
\end{split}\raisetag{3\baselineskip}
\end{equation}

Apply \cref{eqn:plusminus1} and rearrange \cref{eqn:gatherterms}

\begin{equation}
\begin{split}
N_{i,p}'(t) =& \frac{1}{t_{i+p} - t_i} N_{i,p-1}(t)\\
& + \frac{-1}{t_{i+p+1} - t_{i+1}} N_{i+1,p-1}(t) \\ 
& + \frac{p-1}{t_{i+p} - t_i} \left( \frac{t-t_i}{t_{i+p-1}-t_i} N_{i,p-2}(t) + \frac{t_{i+p}-t}{t_{i+p}-t_{i+1}} N_{i+1,p-2}(t) \right)\\
& - \frac{p-1}{t_{i+p+1} - t_{i+1}}\left( \frac{t-t_{i+1}}{t_{i+p}-t_{i+1}} N_{i+1,p-2}(t) + \frac{t_{i+p+1}-t}{t_{i+p+1}-t_{i+2}} N_{i+2,p-2}(t) \right)
\end{split}\raisetag{3\baselineskip}
\end{equation}

We can see that the terms in parentheses now are precisely the relation defined in \cref{eqn:basisgeneral}. Substituting in the left hand side of that equation, we are left with

\begin{equation}
\begin{split}
N_{i,p}'(t) =& \frac{1}{t_{i+p} - t_i} N_{i,p-1}(t)\\
& + \frac{-1}{t_{i+p+1} - t_{i+1}} N_{i+1,p-1}(t) \\ 
& + \frac{p-1}{t_{i+p} - t_i} N_{i,p-1}(t) \\
& - \frac{p-1}{t_{i+p+1} - t_{i+1}} N_{i+1,p-1}(t)
\end{split}\raisetag{3\baselineskip}
\end{equation}

Simplifying, we obtain \cref{eqn:basisderivative}

\[
N_{i,p}'(t) = \frac{p}{t_{i+p} - t_i} N_{i,p-1}(t) - \frac{p}{t_{i+p+1} - t_{i+1}} N_{i+1,p-1}(t)
\]

\end{proof}

\end{appendices}

\end{document}